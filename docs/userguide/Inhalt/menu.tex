\chapter{Menü}
\label{cha:menu}

Das Menü untergliedert sich in eine \nameref{sec:menu_navigation} (Kapitel \ref{sec:menu_navigation}) und verschiedenen Übersichts- bzw Schnellzugriffs- Möglichkeiten (Kapitel \ref{sec:menu_approvedservice}, \ref{sec:menu_qualification}, \ref*{sec:menu_profile}, \ref{sec:menu_logout})

\begin{figure}[h]
 \begin{addmargin}{-0.2\linewidth}
   \centering 
   \includegraphics[width=14cm]{Bilder/view_menu.png}
 \end{addmargin} 
 \caption[Menü ansicht]{DLRG Dienstplan Menü ansicht}
 \label{fig:view_menu}
\end{figure}

\section{Zugeteilte Dienste}
\label{sec:menu_approvedservice}
Darstellung als Schnellübersicht der bestätigten Dienste für den jeweiligen Benutzer.

\noindent (Abbildung \ref{fig:view_menu} \textit{\nameref{fig:view_menu}}, Markierung \textit{1})

\section{Eigene Qualifikationen}
\label{sec:menu_qualification}
Darstellung als Schnellübersicht der zugeteilten Qualifikationen für den jeweiligen Benutzer.

\noindent (Abbildung \ref{fig:view_menu} \textit{\nameref{fig:view_menu}}, Markierung \textit{2})

\section{Profil}
\label{sec:menu_profile}
Schnellzugriff auf das Benutzerprofil.

\noindent (Abbildung \ref{fig:view_menu} \textit{\nameref{fig:view_menu}}, Markierung \textit{3})

\noindent Im Profil kann ein Benutzer seine Daten ändern und auf dem Aktuellsten Stand halten. Wenn in die Eingabe des Passwortes leer bleibt wird dieses nicht geändert. Beim erstmaligen Login empfiehlt es sich hier die Handynummer zu hinterlegen.  

\section{Logout}
\label{sec:menu_logout}
Über den Menüpunkt wird der Benutzer abgemeldet.

\noindent (Abbildung \ref{fig:view_menu} \textit{\nameref{fig:view_menu}}, Markierung \textit{4})

\section{Navigation}
\label{sec:menu_navigation}
Aufgrund einer niedrigen Auflösung oder kleineren Displaygröße wird das Menü auf der Linken Seite nicht dargestellt. Das Menü kann dort über die links oben dargestellten drei Striche ein bzw. ausgeblendet werden.

\noindent Die Navigation enthält bei einem Benutzer folgende Einträge: 
\begin{itemize}
\item \nameref{cha:dashboard} (Kapitel: \ref{cha:dashboard})
\item \nameref{cha:dienste} (Kapitel: \ref{cha:dienste})
\item \nameref{cha:nachrichten} (Kapitel: \ref{cha:nachrichten})
\end{itemize}